%-------------------------------------------------------------------------------------------------------------------------------%
\Section(pCT code Hierarchy Diagram){\texttt{pCT\_code} Hierarchy Diagram}%
[%
    the diagram below shows the hierarchy of \docentry[constdir]{pCT\_code} directories listed and described in the \autohyperlink[type=Subsection]{pCT Code Hierarchy} section, including the clones of the most commonly used GitHub accounts/repositories relevant to pCT (described in the \autohyperlink[type=Section]{GitHub Accounts/Repositories} section).%
]%
%-------------------------------------------------------------------------------------------------------------------------------%
%-------------------------------------------------------------------------------------------------------------------------------%
%-------------------------------------------------------------------------------------------------------------------------------%
%\centering%
\begin{tcbdiagram}
%\centering%
\matrix[matrix of nodes, row sep=4mm,column sep=1.5mm]
{%
&\node(parentdot)[pt]{};\\
&\node(01)[D]{ion};\\
&\node(11)[D]{pCT\_code};\\[-1mm]
\node(20)[pt]{};&\node(21)[pt]{};&&\node(23)[pt]{};\\[2mm]
\node(30)[D]{user\_code};    &&&\node(33)[D]{git};\\[0mm]
\node(40)[D]{$<$username$>$};&&&&&\\[0mm]
\node(vdots)[textonly]{$\vdots$};&&&&&\\[2mm]
\node(50)[pt]{};&\node(51)[pt]{};&&\node(53)[pt]{};&&\node(55)[pt]{};\\[2mm]
\node(60)[D]{BlakeSchultze};&\node(61)[D]{BaylorICTHUS};&&&&\node(65)[D]{pCT-collaboration};\\[-1mm]
&&&\node(73)[pt]{};&\node(74)[pt]{};&\node(75)[pt]{};&\node(76)[pt]{};&\node(77)[pt]{};\\[2mm]
\node(80)[F]{pCT\_Reconstruction};&\node(81)[F]{pCT\_Reconstruction};&&\node(83)[F]{pCT\_Documentation};&\node(84)[F]{pCT\_Tools};&\node(85)[F]{Kodiak\_Reconstruction};&\node(86)[F]{pct-recon-copy};&\node(87)[F]{Preprocessing};\\[1mm]
};
\graph[use existing nodes, edge label, edges=thick]
{
(parentdot)->(01)->(11)--(21);
(vdots)<-(40)<-(30)<-(20)--(23)->(33)--(53)--(50)->(60)->(80);
(53)--(55)->(65)--(75)->(85);
(83)<-(73)--(77)->(87)
(51)->(61)->(81);
(74)->(84);
(76)->(86);
};
\end{tcbdiagram}
%*******************************************************************************************************************************%
%*******************************************************************************************************************************%
%*******************************************************************************************************************************%
\endinput 